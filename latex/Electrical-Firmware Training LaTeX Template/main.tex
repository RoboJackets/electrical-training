\documentclass{article}
\usepackage{geometry}
\usepackage{flafter}
\geometry{letterpaper, portrait, margin=1in}

\usepackage{hyperref}
\hypersetup{
    colorlinks=true,
    linkcolor=black,
    filecolor=magenta,
    urlcolor=blue,
}

\usepackage{graphicx}
\graphicspath{ {images/} }

\usepackage{tcolorbox}
\usepackage{textcomp}
\usepackage{gensymb}
\usepackage{indentfirst}

\newcommand{\ans}{$\rule{1.5cm}{0.15mm}$}

\title{RoboJackets Electrical/Firmware Soldering Training Guide}
\author{Bernardo Perez and Andrew Roach}
\date{\today\\v1.0}

\begin{document}
\maketitle{}
\setcounter{tocdepth}{2}
\tableofcontents
\pagebreak

%Everything below is for you to edit. Code above sets up the general formatting for the document

\section{Background}
Insert information about the lecture topic, how this is applicable to RoboJackets, introduce the lab.

\section{Objective}
\subsection{Task 1}
\begin{itemize}
    \item List the tasks/subtasks that will be completed, in order of difficulty if applicable.
\end{itemize}
\subsection{Task 2}
\begin{itemize}
    \item List the tasks/subtasks that will be completed, in order of difficulty if applicable.
\end{itemize}
\section{Materials}
\begin{itemize}
	\item List the materials that they need to complete the lab, hyperlink datasheets.
\end{itemize}

\section{Relevant Information}
\subsection{Topic 1}
Here put in any technical information that would necessary in order to help people complete the lab
If applicable include any information to help people get started, especially on the first task.
\subsection{Topic 2}
Here put in any technical information that would necessary in order to help people complete the lab
If applicable include any information to help people get started, especially on the first task.

\section{Guided Lab (mainly for EAGLE weeks)}
\subsection{General Step 1}
\subsubsection{More Specific Sub-Step}
\begin{itemize}
    \item List things if applicable, add pictures (don't forget captions)!
\end{itemize}

\section{Troubleshooting}
Put any information that point people in the right direction for common mistakes (Ex: if something is not receiving power, check for short, check VCC and GND wires on breadboard).

%Example figure
\begin{figure}[ht]
	\center
	\includegraphics[width=0.1\textwidth, keepaspectratio]{images/robobuzz.png}
	\caption{Pictures are nice! Remember to add a nice caption to the image!}
	\label{fig:robobuzz}
\end{figure}

\end{document}